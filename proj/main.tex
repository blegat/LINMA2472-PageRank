\documentclass{article}

\usepackage[utf8x]{inputenc}
\usepackage[frenchb]{babel}
\usepackage[T1]{fontenc}
\usepackage{lmodern}
\usepackage{fullpage}
\usepackage{graphicx}
\usepackage{epstopdf}
\usepackage{caption}
\usepackage{subcaption}
\usepackage{multirow}

% Math symbols
\usepackage{amsmath}
\usepackage{amssymb}
\usepackage{amsthm}

% Numbers and units
\usepackage{siunitx}

\DeclareMathOperator{\newdiff}{d} % use \dif instead
\newcommand{\dif}{\newdiff\!}
\newcommand{\fpart}[2]{\frac{\partial #1}{\partial #2}}
\newcommand{\ffpart}[2]{\frac{\partial^2 #1}{\partial #2^2}}
\newcommand{\fdpart}[3]{\frac{\partial^2 #1}{\partial #2\partial #3}}
\newcommand{\fdif}[2]{\frac{\dif #1}{\dif #2}}
\newcommand{\ffdif}[2]{\frac{\dif^2 #1}{\dif #2^2}}
\newcommand{\constant}{\ensuremath{\mathrm{cst}}}

\DeclareMathOperator{\argmax}{argmax}
\newcommand{\1}{\mathbf{1}}

\author{Benoît Legat}
\title{PageRank optimization}

\begin{document}

\maketitle

\section{Reminder}
We will adopt the transposed notation due to the fact that all papers do so.

So
\[ \pi_{k+1}^T = \pi_k^T cP + t^T(1-c) \]
or
\[ \pi_{k+1}^T = \pi_k^T (cP + \1 t^T(1-c)) \]
and at the convergence
\[ \pi^T = \pi^T (cP + \1 t^T(1-c)) \]
so
\[ \pi^T = (1-c)\1 t^T(I - cP)^{-1} \]
where $(1 - cP)^{-1} \triangleq Z$ as we will see later.

\section{The effect of adding links}
\cite{avrachenkov2004decomposition} show that adding a link
from $i$ to $j$ always increase the PageRank of $j$.
The worst case for a page is having no inlinks which gives a PageRank of $(1-c)/n$.

It analyses the decomposition of the web in communities,
set of pages that does not link to each other (because they are of a difference subject for example).
This is excellent for parallelizing the computation of PageRank.
He then proves that a community has no interest in adding link to another community,
in will decrease its PageRank.

This is an example that show that adding an outlink for a page $p$ can decrease the PageRank of $p$.
However, it can also increase the PageRank of a page.
Consider for example the case where $p$ links to a page of another community.
Adding a link from $p$ to a page that links to $p$ can decrease the number of random surfers that go to this other community and increase the PageRank of $p$.

This gives the intuition to a new information we need about pages.
When a random surfer is at $q$, it will walk randomly from $q$ and will eventually zap to a random page (which might be $p$ with probability $1/n$).
Before zapping, this random surfer might already pass through $p$.
We can define $z_{pq}$ as the expected number of times that a random surfer starting at $q$ pass through $p$ (counted $q$ if $q=p$) before zapping.
$p$ should rather have outlinks to pages $q$ that have high $z_{pq}$.

It is therefore not so surprising to discover that $z_{pq}$ is actually $e_p^T(I-\beta P)^{-1}e_q$.
\cite{avrachenkov2006effect} discovers all that and concludes by giving the optimal outlink strategy for a webpage which is having only one outlink pointing to $q^* = \argmax_{p \neq q} z_{pq}$.
Of course, the best strategy is to only have a self loop but that would mean that $p$ is a spider trap which can be detected.
If the search engine allows self-loop the optimal strategy is to have 2 links, a self-loop and a link to $q^*$.

\section{Choice of inlinks (aka backlinks)}
A first question might be to try to optimize the PageRank of a page
by adding $k$ new link.

\cite{olsen2010maximizing} shows that this problem has no fully polynomial time scheme (FPTAS) if $P \neq NP$ and also show that it is W[1]-hard.

\cite{olsen2010constant} compares the naïve algorithm for this problem and a $r$-Greedy algorithm he develops.
He shows that the naïve algorithm does not perform very well and he shows a lower bound for the pagerank of its page which is a constant times the best pagerank possible. It is
\[ \pi_p^G \geq \pi_p^o(1 - \beta^2)(1 - 1/e) \]
where $p$ is the page for which we try to maximize the PageRank,
$\pi_p^G$ is the PageRank obtained by its $r$-Greedy algorithm, $\pi_p^o$ is the best PageRank that could be optained, $\beta$ is therelaxation factor and $e$ is the neperian constant.
For the usual value $\beta = 0.85$, this constant is approximatively $0.175$ which may seems not that good but is not terrible either.

\bibliographystyle{plain}
\bibliography{biblio}

\end{document}
