\documentclass{article}

\usepackage[utf8x]{inputenc}
\usepackage[english]{babel}
\usepackage[T1]{fontenc}
\usepackage{lmodern}
\usepackage{fullpage}
\usepackage{graphicx}
\usepackage{epstopdf}
\usepackage{caption}
\usepackage{subcaption}
\usepackage{multirow}
\usepackage{todonotes}

% Math symbols
\usepackage{amsmath}
\usepackage{amssymb}
\usepackage{amsthm}

% Numbers and units
\usepackage{siunitx}

\DeclareMathOperator{\newdiff}{d} % use \dif instead
\newcommand{\dif}{\newdiff\!}
\newcommand{\fpart}[2]{\frac{\partial #1}{\partial #2}}
\newcommand{\ffpart}[2]{\frac{\partial^2 #1}{\partial #2^2}}
\newcommand{\fdpart}[3]{\frac{\partial^2 #1}{\partial #2\partial #3}}
\newcommand{\fdif}[2]{\frac{\dif #1}{\dif #2}}
\newcommand{\ffdif}[2]{\frac{\dif^2 #1}{\dif #2^2}}
\newcommand{\constant}{\ensuremath{\mathrm{cst}}}
\newcommand{\bigoh}{\mathcal{O}}

\DeclareMathOperator*{\argmin}{arg\,min}
\DeclareMathOperator*{\argmax}{arg\,max}
\newcommand{\1}{\mathbf{1}}

% Theorem and definitions
\theoremstyle{definition}
\newtheorem{mydef}{Definition}
\newtheorem{mynota}[mydef]{Notation}
\newtheorem{myprop}[mydef]{Property}
\newtheorem{myrem}[mydef]{Remark}
\newtheorem{myform}[mydef]{Formula}
\newtheorem{mycorr}[mydef]{Corrolary}
\newtheorem{mytheo}[mydef]{Theorem}
\newtheorem{mylem}[mydef]{Lemma}
\newtheorem{myexem}[mydef]{Example}
\newtheorem{myineg}[mydef]{Inequality}



\author{Benoît Legat \and Quentin Laurent}
\title{A algorithm proposal for pagerank optimization}

\begin{document}

\maketitle

\section{Introduction and notations}
The importance of a node in a graph is often measured by its pagerank. One of the application domain is the ordering of pages in the world wide web. It is legitimate for the owner of a particular site to seek the optimal pagerank for his webpage(s). We will examine which alternatives are to be considered in order to get an optimal solution for the sums of the pageranks of a set of pages.

We formerly introduced pagerank, and in \cite{leskovec}, we saw there ways of dealing with dangling nodes or spider traps, i.e. nodes or groups of nodes without outlinks. Since those nodes are very likely to be removed from the page order, we will see how to maximize a pagerank while still being part of the same strongly connected component, i.e. not being detected as a spider trap.

It is reasonable to suppose that a page owner only has control over the outlinks of his pages. The notation we are going to introduce is mainly issued from \cite{de2008maximizing}.
Let $\mathcal{G} = (\mathcal{N},\mathcal{E})$ be the webgraph, with a set of nodes  $\mathcal{N} = \{1,2,...,N\}$ and a set of nodes $\mathcal{E}\subseteq \mathcal{N}\times \mathcal{N}$.

By $j \leftarrow i$ we denote that $(i,j)\in \mathcal{E}$, $i$ is a parent of $j$ and $j$ is a child of $i$. 

We define the set of the pages of our website as  $\mathcal{P} \subset \mathcal{N}$. Subsequently, we define :
\begin{eqnarray*}
\mathcal{E}_{\mathcal{P}} & = & \{(i,j)\in \mathcal{E} : i,j \in \mathcal{P}\} \text{ the set of internal links of }\mathcal{P} \text{ (controlled)} \\
  \mathcal{E}_{\text{out}(\mathcal{P})} & = & \{(i,j)\in \mathcal{E} : i \in \mathcal{P},j \in \overline{\mathcal{P}}\} \text{ the set of outlinks of }\mathcal{P} \text{ (controlled)}\\
  \mathcal{E}_{\text{in}(\mathcal{P})} & = & \{(i,j)\in \mathcal{E} : i \in \overline{\mathcal{P}},j \in \mathcal{P}\} \text{ the set of inlinks of }\mathcal{P} \\
\mathcal{E}_{\overline{\mathcal{P}}} & = & \{(i,j)\in \mathcal{E} : i,j \in \overline{\mathcal{P}}\} \text{ the set of external links} 
\end{eqnarray*}
We also make the following assumptions\\
\textit{Assumption 1 :} Every page of the website must have access to $\overline{\mathcal{P}}$.\\
\textit{Assumption 2 :} $\forall i\in \mathcal{N}$, the degree $d_i$ of $i$ is non-zero.\\
Let's now define the pagerank score (since most article use this notation we found it more convenient to choose it). It is the transpose of the notation used in \cite{leskovec}.
We define the stochastic matrix $P$ such that :
$$ P_{ij} = \left\lbrace \begin{array}{ll}
d_i^{-1} & \text{if }(i,j)\in \mathcal{E} \\
0 & \text{otherwise}
\end{array}
\right.$$
Let also $0<c<1$ be a damping factor and $z$ be a positive stochastic personalization vector($z^T \mathbf{1} = 1$ and $z_i>0 \:\forall i = 1,...,n$).

Hence the pagerank vector satisfies
\[ \pi^T = \pi^T (cP + \1 z^T(1-c)) \]
so
\[ \pi^T = (1-c)\1 z^T(I - cP)^{-1} \]
where $(1 - cP)^{-1} \triangleq Z$ as we will see later.
\section{Pagerank of a set of webpages}
Our goal here is to maximize the sum of the pageranks of a set of webpages, represented by the set $\mathcal{P}$. This is given by
$$\pi^Te_{\mathcal{P}} = \sum_{i\in \mathcal{P}}\pi_i$$
$$\pi^Te_{\mathcal{P}} = (1-c)z^Tv$$
with $e_{\mathcal{P}}$ is the vector with a $1$ in the entry of $\mathcal{P}$ and a $0$ elsewhere, and $v \triangleq Ze_{\mathcal{P}}$.

As mentioned in \cite{de2008maximizing}, the vector $v$ can be viewed as the expected number of visits to the set $\mathcal{P}$ before zapping (with probability $(1-c)$).
It can be explained very easily by looking at its definition
\[ v = (I - cP)^{-1} e_{\mathcal{P}}. \]
This implies that
\[ v = e_{\mathcal{P}} + cPv \]
so for $i \in \mathcal{P}$, we have
\[ v_i = 1 + c \sum_{j \leftarrow i} \frac{v_j}{d_i} \]
so $v_i$ is 1 (because it is at $i \in \mathcal{P}$ which counts as 1 visit in $\mathcal{P}$ without zapping)
added with the probability that it doesn't zap, i.e. $c$, multiplied by the sum over its childs $j$ of the number of visit without zapping if it goes at $j$ multiplied
by the probability that it goes at $j$, i.e. $1/d_i$.

For $i \in \mathcal{P}$, we have
\[ v_i = c \sum_{j \leftarrow i} \frac{v_j}{d_i} \]
which has the same interpretation except that we do not add 1 because we are not visiting a node of $\mathcal{P}$.

We will see that it is much more convenient to work with $v$ than with $\pi$.
$\pi$ can be viewed as a baked cookie and $v$ as the cookie dough which is a lot easier to manipulate
while $\pi$ is just good to eat.

\section{The effect of adding links}
\cite{avrachenkov2004decomposition} shows that adding a link
from $i$ to $j$ always increases the PageRank of $j$.
The worst case for a page is to have no inlinks which gives it a PageRank of $(1-c)/n$.

It analyses the decomposition of the web in communities, i.e.
sets of pages that do not link to each other (because they treat of a different subject for example).
This is excellent for parallelizing the computation of PageRank.
He then proves that a community has no interest in adding a link to another community, since it will decrease its PageRank.

This is a simple example showing that adding an outlink from a page $p$ can decrease the PageRank of $p$.
However, it can also increase it.
Consider for example, two community: the community of pages about cookies and the community of pages about cosmology.
If a page $p$ that talks about cosmology makes a metaphor about the cooking of cookies, it might link to a page of the other community.
This will leak random surfers from its community and will likely decrease its pagerank.
If the writer of this article get a bit more serious and add another reference to a page in cosmology, the probability to follow the link that leads to the art of cooking cookies decreases which increases the pagerank of $p$.
This shows that the 2 cases are possible.

This gives the intuition to a new information we need about pages.
When a random surfer is at $q$, it will walk randomly from $q$ and will eventually zap to a random page (which might be $p$ with probability $1/n$).
Before zapping, this random surfer might already pass through $p$.
We can define $z_{pq}$ as the expected number of times that a random surfer starting at $q$ passes through $p$ (counted $q$ if $q=p$) before zapping.
$p$ should rather have outlinks to pages $q$ that have high $z_{pq}$.
This looks like the vector $v$ we have defined earlier !
This is actualy the $v_q$ for $\mathcal{P} = \{p\}$, i.e. $z_{pq} = e_q^T(I-cP)^{-1}e_{\{p\}}$.

\cite{avrachenkov2006effect} discovers all that and concludes by giving the optimal outlink strategy for a webpage,
which has only one outlink pointing to $q^* \in \argmax_{p \neq q} z_{pq}$.
Of course, the best strategy is to only have a self loop but that would mean that $p$ is a spider trap which can be detected.
If the search engine allows self-loop the optimal strategy is to have 2 links, a self-loop and a link to $q^*$.
If $|\argmax_{p \neq q} z_{pq}| > 1$, there is several possibilities which are necessarily all optimal.



\section{Best link structure for a set of pages}
The best link structure has already been discussed for a single node.
In \cite{de2008maximizing} it is demonstrated that the subgraph of a set of pages which maximizes the PageRank of a set $\mathcal{P}$, denoted $\pi^T e_{\mathcal{P}}$ has a particular structure.

\begin{mytheo}
  It is stated that for $\mathcal{E}_{\text{in}(\mathcal{P})}$, $\mathcal{E}_{\overline{\mathcal{P}}}$ given, if  $\pi^T e_{\mathcal{P}}$ is maximal under assumption 1 and 2, then there exists a permutation of the indices such that 
$\mathcal{P} = \{1,2, \hdots, n_{\mathcal{P}}\}$,
$$ v_1 > ... > v_{n_P}>v_{n_{\mathcal{P}}+1} \geq ... \geq v_n$$
and $\mathcal{E}_{\mathcal{P}}$ and $\mathcal{E}_{\text{out}(\mathcal{P})} $ have the following structure :
$$\mathcal{E}_{\mathcal{P}} = \{(i,j) \in \mathcal{P}\times \mathcal{P} : j \leq i \text{ or }j = i+1 \}$$
$$\mathcal{E}_{\text{out}(\mathcal{P})} = \{(n_{\mathcal{P}},n_{\mathcal{P}}+1)\}$$
We also know from \cite{de2008maximizing} that $n_\mathcal{P}+1 \in \mathcal{V}$ where
\[ \mathcal{V} = \argmax_{j \in \overline{\mathcal{P}}} v_j. \]
We also know from \cite{de2008maximizing} that $\mathcal{V} \subseteq \{j \in \overline{\mathcal{P}} | \exists i \in \mathcal{P} : i \leftarrow j\}$, the parents of $\mathcal{P}$.
\label{thm:optstruct}
\end{mytheo}

If we want to use this structure in practice, a problem remains.
We do not know in which order the pages need to be in the chain
nor which is the sinking page (even if we know that it points to a parent of $\mathcal{P}$).

There is a conjecture at the end of \cite{de2008maximizing}.
Instead of simply proving the conjecture, we will prove a generalization.
\begin{mytheo}[Partition of the chain]
  \label{theo:partition}
  Let $P_1,P_2 \subseteq P$ be a partition of $P$ such that
  $i \in P_1$ if $i$ has an in-link from $\bar{P}$ and $i \in P_2$
  otherwise.
  If $z_i = z_j$ for all $i,j \in P$,
  then with the optimal link structure,
  $v_i > v_j$ for all $(i,j) \in P_1 \times P_2$.
  In other words, in the chain there is first all pages of
  $P_1$ then all pages of $P_2$.
  \begin{proof}
    Let's suppose that their is $(i,j) \in P_1 \times P_2$
    such that $v_i < v_j$ ($v_i = v_j$ is impossible (see theorem \ref{thm:optstruct})).

    Then since $z_i = z_j$, we can simulate the situation of swapping $i$ and $j$ in the chain
    by giving the inlinks of $i$ to $j$.
    Using \cite[theorem~5]{de2008maximizing}, we see that we increase the PageRank each time we give one of the inlinks since
    we have each time
    \[ \delta^Tv = \frac{v_j - v_i}{d} > 0 \]
    where $d$ is the out-degree of the parent of $i$ for which we change de destination of the link.
  \end{proof}
\end{mytheo}

\cite{de2008maximizing} says
``If this conjecture was true we could also ask if the node $j \in \mathcal{P}$ belongs to $\mathcal{V}$ if $j$ is such that
$(j, i) \in \mathcal{E}_{\text{in}(\mathcal{P})}$ where $i \in \argmax_k v_k$ ''.
Sadly, this is not so simple. \todo{find a counter example}\\
As we will see, the first page of the forward chain could be there not
because it has a parent in $\mathcal{V}$ but because it has many parents with high $v$ even if they are not in $\mathcal{V}$.

\subsection{Maximizing the pagerank without self-links}
In this case the optimal structure is simply modified by substracting the self links of the optimal structure described in theorem \ref{thm:optstruct}. The theorem 11 and 12 of \cite{de2008maximizing} generalize very easily in this case.

\subsection{Optimal structure with several external outlinks}


\section{Optimal order}
\subsection{Optimal order with no leaking node}
\label{sec:no_leak}
Let's now have a prelude, showing how we could easily choose the order of the pages in the chain if there were
no leaking pages.
In that case, our set of pages would of course be a spider trap and the optimal linkage strategy wouldn't be a chain, but this derivation is similar
to the one with a leaking page and helps to its understanding.\\
We define the basic absorbing graph relative to the set $\mathcal{P}$ of our graph as $\mathcal{G}_0=(\mathcal{N},\mathcal{E}^0)$, where $\mathcal{E}_{\text{out}(\mathcal{P})}^0 = \emptyset$,$\mathcal{E}^0_{\mathcal{P}}=\{(i,i):i\in \mathcal{P}\},\mathcal{E}_{\text{in}(\mathcal{P})}^0 = \mathcal{E}_{\text{in}(\mathcal{P})}$ and $ \mathcal{E}^0_{\overline{\mathcal{P}}}=\mathcal{E}_{\overline{\mathcal{P}}}$.\\
We now try to find the relation between the graph with the forward chain and the basic absorbing graph.
Let $v^{0_i}$ be the usual $v$ for the basic absorbing graph relative to $i$,
i.e. $v^{0_i} = (I - cP)^{-1}e_i$ where the $i$th line of $P$ is $e_i^T$.
In order to do that, we will introduce $p_{ijk}$ which is the probability that a random surfer starting at $j$
will arrive at $i$ after following $k$ links before zapping.
We have naturally $p_{iik} = c^k$ since the only out-link of $i$ leads to $i$ in its basic absorbing graph.
Therefore
\[ v_i^{0_i} = \sum_{k=0}^\infty c^k = \lim_{k \to \infty} \frac{1-c^k}{1-c} = \frac{1}{1-c}. \]
For $j \neq i$, we can reuse the value $v_i^{0_i}$ we have just found one we arrive at $i$.
\[ v_j^{0_i} = \sum_{k=0}^\infty p_{ijk} v_i^{0_i} = \frac{1}{1-c} \sum_{k=0}^\infty p_{ijk}. \]

For our forward chain,
once we arrive at a page $i \in P$, we use $v_i$.
So for $j \in \bar{P}$
\begin{align*}
  v_j & = \sum_{i \in P} \sum_{k = 0}^\infty p_{ijk} v_i\\
      & = \sum_{i \in P} v_i \sum_{k = 0}^\infty p_{ijk}\\
      & = \sum_{i \in P} v_i (1-c) v_j^{0_i}
\end{align*}
and therefore
\begin{align*}
  z^T v
  & = \sum_{i \in P} z_i v_i + \sum_{j \in \bar{P}} z_j v_j\\
  & = \sum_{i \in P} z_i v_i + \sum_{j \in \bar{P}} z_j\sum_{i \in P} v_i (1-c) v_j^{0_i}\\
  & = \sum_{i \in P} z_i v_i + \sum_{i \in P} v_i \sum_{j \in \bar{P}} (1-c) z_j v_j^{0_i}\\
  & = \sum_{i \in P} v_i \left(z_i + \sum_{j \in \bar{P}} (1-c) z_jv_j^{0_i}\right)\\
  & = w^T v_P
\end{align*}
where $w$ is defined as expected.
We know from \cite{de2008maximizing} that the order of $v_i$ (which is strict) is the same that
the order of the pages in the chain, i.e.
\[ v_{n_P} < v_{n_P-1} < \cdots < v_2 < v_1 \]
where $1$ is the leaking page, $v_2$ is the page before it, ...

Choosing the order maximizing this scalar product is trivial using the ``Rearrangement inequality''.
The order of $w$ gives us therefore immediately the order of our pages in the forward chain.
We can obtain $w$ solving the $n_P$ system $(I - cP^{0_i})v^{0_i} = e_i$.
This is far better than running $n_P!$ times PageRank.

\subsection{Optimal order}
Let $l$ be the leak, i.e. $(1,l) \in E$ and $l \in \bar{P}$.
Wlog, let $l = n_{P}+1$.
Let $P_P$ be the part of $P$ with only the pages $1, \ldots, n_P$.
In section~\ref{sec:no_leak},
\[ (I - cP)v = e_P \]
implied that
\[ (I - cP_P)v_P = \1 \]
since
\[ P =
  \begin{pmatrix}
    P_P & 0\\
    * & *
  \end{pmatrix}.
\]
where
\[
  P_P =
  \begin{pmatrix}
    (n+1)^{-1} & (n+1)^{-1} & \cdots & (n+1)^{-1} & (n+1)^{-1}\\
    n^{-1} & n^{-1} & \cdots & n^{-1} & n^{-1}\\
           & (n-1)^{-1} & \cdots & (n-1)^{-1} & (n-1)^{-1}\\
           & & \ddots & \vdots & \vdots\\
    & & & 1/2 & 1/2
  \end{pmatrix}
\]

Now, the $n_P$ first lines of $P$ are
\[
  \begin{pmatrix}
    (n+1)^{-1} & (n+1)^{-1} & \cdots & (n+1)^{-1} & (n+1)^{-1} & 1 & 0 & \cdots & 0\\
    n^{-1} & n^{-1} & \cdots & n^{-1} & n^{-1} & 0 & 0 & \cdots & 0\\
           & (n-1)^{-1} & \cdots & (n-1)^{-1} & (n-1)^{-1} & 0 & 0 & \cdots & 0\\
           & & \ddots & \vdots & \vdots & \vdots & \vdots & & \vdots \\
           & & & 1/2 & 1/2 & 0 & 0 & \cdots & 0
  \end{pmatrix}.
\]
and therefore
\[ (I - cP_P)v_P = \1 + c e_1 v_l. \]

But we know that
\[ v_l = (1-c) \sum_{i \in P} v_i v_l^{0_i}. \]
Noting
\[ v_l^0 =
  \begin{pmatrix}
    v_l^{0_1} \\ v_l^{0_2} \\ \vdots \\ v_l^{0_{n_P}}
  \end{pmatrix}
\]
we have
\[ v_l = (1-c) {v_l^0}^T v_P \]
and consequently
\[ (I - cP_P)v_P = \1 + c (1-c) e_1 {v_l^0}^T v_P \]
or
\[ (I - cP_P - c (1-c) e_1 {v_l^0}^T)v_P = \1. \]

Using Sherman-Morrison, we have
\begin{align*}
  (I - cP_P - c (1-c) e_1 {v_l^0}^T)^{-1} & = (I - cP_P)^{-1} + (I - cP_P)^{-1}e_1 c(1-c) \frac{{v_l^0}^T(I-cP_P)^{-1}}{1 - c(1-c){v_l^0}^T(I-cP)^{-1}e_1}\\
  (I - cP_P - c (1-c) e_1 {v_l^0}^T)^{-1}\1 & = h + g c(1-c) \frac{{v_l^0}^Th}{1 - c(1-c){v_l^0}^Tg}
\end{align*}
defining $h,g$ appropriatelly.

We now have to maximize
\[ w^T \left(h + g c(1-c) \frac{{v_l^0}^Th}{1 - c(1-c){v_l^0}^Tg}\right) \]
where $h$ and $g$ can be precomputed.
The only problem is that we have to choose the permutation $\sigma$ that we apply to $w$ and $v_l^0$ (the same !).
For $v_l^0$, if there is 0 elements, we can have less than $n_P!$ (and the order of the others is trivial to find with rearrangement like before).
In the worst case we still have $n_P!$ but we do not have to compute the pagerank each time but only evaluating this expression which is $\bigoh(n_P)$ !

More generally, since $g$ and $h$ are in the same order \todo{prove it} and because we should maximize both ${v_l^0}^Tg$ and ${v_l^0}^Th$ to have the maximum expression (since $1 > c(1-c){v_l^0}^Tg$\todo{prove it}),
if there is $i_1, i_2$ for which $w_{i_1}, w_{i_2}$ are in the same order than ${v_l^0}_{i_1}, {v_l^0}_{i_2}$,
we can deduce the order of $i_1, i_2$ in $g$ and $h$ so in practice we will be able to do a lot better than $n_P!$ using this trick too.


The theorem~\ref{theo:partition} is a simple consequence of this.
If $z_{i_1} = z_{i_2}$, $i_2$ has no in-link from $\bar{P}$ and $i_1$ has at least one,
then we have that
${v_l^0}_{i_2} = v_l^{0_{i_2}} = 0 < v_l^{0_{i_1}}$ and $w_{i_2} = z_{i_2} < z_{i_1} + \cdots = w_{i_1}$
which implies that $i_1$ is prior to $i_2$ in the forward chain.

\section{Adding new inlinks}
In the case where we can add a link in a webpage, such as a forum or any other accessible page, we include the webpage in the controlled set $\mathcal{P}$.\\
Intuitively, doing so should result in a pagerank rise of the target set, although it is not always the case.\\
%TODO : example
A first question might be to try to optimize the PageRank of a page
by adding $k$ new link.

\cite{olsen2010maximizing} shows that this problem has no fully polynomial time scheme (FPTAS) if $P \neq NP$ and also show that it is W[1]-hard.

\cite{olsen2010constant} compares the naïve algorithm for this problem and a $r$-Greedy algorithm he develops.
He shows that the naïve algorithm does not perform very well and he shows a lower bound for the pagerank of its page which is a constant times the best pagerank possible. It is
\[ \pi_p^G \geq \pi_p^o(1 - \beta^2)(1 - 1/e) \]
where $p$ is the page for which we try to maximize the PageRank,
$\pi_p^G$ is the PageRank obtained by its $r$-Greedy algorithm, $\pi_p^o$ is the best PageRank that could be optained, $\beta$ is the relaxation factor and $e$ is the neperian constant.
For the usual value $\beta = 0.85$, this constant is approximatively $0.175$ which may seems not that good but is not terrible either.\\
If the target set is a singleton, adding a link towards this singleton does increase its pagerank. But is it the optimal strategy? It would indeed appear that the best thing we can do by adding a link from a site is add a link from the site to $argmax_{v_k}$\\ \todo[inline]{Proof}
When two pages $i$ and $j$ have similar expected number of steps before zapping $v_i$ and $v_j$, we can guess that adding a link from the page with the highest pagerank should yield a better result. How can we prove it? 

\section{Incomplete information and approximation of z}
In order to be able to apply those results in the real world, we need to be able to compute the inverse matrix $(I-cP)e_{\mathcal{P}}$. However, this is not such an easy thing to do, given the actual computational power and storage needed. In this section we will partition the set of nodes, $V$ in two sets $\mathcal{L}_1$ and $\mathcal{L}_2$. We are here only interested in the elements $z_{ij}$ of $(I-cP)$ so that $i,j \in \mathcal{L}_1$. \\
\\
Coming soon... (or not)

\bibliographystyle{plain}
\bibliography{biblio}

\end{document}
